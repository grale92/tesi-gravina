\chapter{Implementazione}

Il presente capitolo descrive l’implementazione concreta delle soluzioni descritte fino a questo momento, evidenziando le tecnologie utilizzate e le modalità di integrazione tra i diversi moduli.

In particolare, vengono descritte le principali componenti implementative della soluzione custom basata su tecniche di Named Entity Recognition, nonché le modalità di utilizzo della soluzione cloud-based a fini di confronto. Il capitolo si concentra sugli elementi rilevanti per la realizzazione del prototipo, tralasciando i dettagli implementativi di basso livello, e pone l’attenzione sugli aspetti che incidono maggiormente sull'architettura progettuale.

\section{Tecnologie utilizzate}

\subsection{PyMuPDF}

Per l'estrazione del contenuto testuale dei documenti PDF nativi è stata utilizzata la libreria \textbf{PyMuPdf} \cite{pymupdf_docs}, una trasposizione in \textit{Python} del motore di parsing e rendering open source \textit{MuPDF}, progettato per l'elaborazione efficiente di documenti in formato PDF, XPS ed EPUB.

PyMuPDF consente di accedere direttamente alla rappresentazione interna del documento, permettendo di accedere alle sue componenti fondamentali, secondo il livello di granularità richiesto dall'applicativo. In particolare la libreria offre i seguenti livelli gerarchici di rappresentazione:

\begin{itemize}
    \item \textbf{Document}: restituisce la rappresentazione del file PDF nel suo complesso
    \item \textbf{Pages}: restituisce una rappresentazione in una sequenza ordinata di pagine
    \item \textbf{Blocks}: restituisce le porzioni logiche di contenuto (testo, immagini, disegni)
    \item \textbf{Lines / Spans}: restituisce unità testuali più finiti, contenenti stringhe di caratteri con attributi di stile
\end{itemize}

\begin{figure}[H]
    \centering
    \includegraphics[width=0.9\textwidth]{figures/pymupdf_structure.png}
    \caption{Esempio della struttura gerarchica generata da PyMuPDF}
\end{figure}

Oltre alla rappresentazione testuale, PyMuPDF consente di accedere a informazioni aggiuntive relative al layout e metadati del documento, come coordinate spaziali dei blocchi testuali, font e dimensione dei caratteri. Queste informazioni possono essere utilizzate per migliorare l'interpretazione del contenuto, ad esempio distinguere intestazioni, tabelle o sezioni ricorrenti all'interno del testo.

La libreria si denota per elevata velocità di elaborazione grazie al motore su cui è basata, tuttavia non fornisce nativamente strumenti per l'interpretazione semantica delle strutture testuali estratte. Di conseguenza, PyMuPDF viene utilizzata come componente di basso livello all'interno di una pipeline più strutturata e costituisce l'input per successive fasi di analisi semantica e contestuale.

\subsection{spaCy}

\subsection{FastAPI}

\subsection{Google Cloud Document AI}

\section{Implementazione del modello NER}

[TODO]

\section{Fine-tuning con Google Cloud}

[TODO]

\section{Integrazione con l’applicazione esistente}

[TODO]