\chapter{Contesto e Analisi Funzionale}

1.1 L'applicativo per il controllo di idoneità dei materiali

L'applicativo oggetto del progetto di tesi, inquadrato nell'ambito fashion, viene utilizzato dall'azienda per gestire il flusso del controllo di idoneità dei materiali per la produzione di capi di abbigliamento. Nel dettaglio, ogni capo da produrre deve avere una lunga serie di conformità di carattere fisico, chimico, e non solo, che vengono certificate tramite dei test di laboratorio. Questi test vanno effettuati non solo sul prodotto finito, che nell'ambito viene denominato \textit{modello}, ma anche sulle singole componenti di un capo, denominate \textit{articoli}, alcuni esempi di queste componenti possono essere tessuti, bottoni, cerniere, fodere. Nella struttura gerarchica dell'azienda, ad ogni modello sono associati uno o più articoli che lo compongono, ed un articolo può essere presente in più di un modello.

L'applicativo si configura quindi come un repository che permette di gestire, a partire dall'anagrafica dei modelli e degli articoli:
- L'anagrafica dei test che ogni modello / articolo deve superare prima di passare in produzione
- La struttura gerarchica che associa articoli a modelli, in modo da poter avere una visione completa dello stato di avanzamento dei test per quello che diventerà un modello finito
- Il caricamento e lo storage dei documenti di report dei test di laboratorio
- L'associazione dei documenti di report agli articoli / modelli

Gli utenti dell'applicazione non sono solo quindi i dipendenti dell'azienda di abbigliamento, ma viene fornito l'accesso anche ai laboratori stessi (che generalmente sono entità esterne), i quali si occupano del caricamento dei report, questo è un punto cruciale per quella che sarà la problematica affrontata.

1.1.1 Il flusso di caricamento dei documenti di report

1.2 



