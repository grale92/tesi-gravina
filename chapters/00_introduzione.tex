\chapter*{Introduzione}
\addcontentsline{toc}{chapter}{Introduzione}

Negli ultimi anni la crescente digitalizzazione dei processi aziendali ha reso sempre più centrale la questione della gestione automatica dei documenti e delle informazioni in essi contenute. In numerosi contesti applicativi, in particolar modo nei settori industriali e manifatturieri, una quantità significativa di dati continua a essere prodotta sotto forma di documenti cartacei non strutturati o semi-strutturati, come report tecnici, certificazioni di conformità, schede di prodotto e risultati di prove di laboratorio. Tali documenti spesso necessitano di un adeguato processo di estrazione e normalizzazione dei dati prima di poter fornire un reale valore aggiunto in un sistema informativo.

L'estrazione automatica del testo da documenti rappresenta un primo passo fondamentale verso l'automazione dei processi di analisi e gestione dei contenuti documentali. A questo scopo, a seconda della tipologia di input, è necessario individuare lo strumento più appropriato, utilizzando varie tecniche di \textit{Text Extraction}, che possono richiedere anche l'ausilio di sistemi di OCR (\textit{Optical Character Recognition}) qualora il testo da estrarre non sia in un formato digitale nativo (è il caso di scansioni o immagini).

Tuttavia, una semplice estrazione del testo non è sufficiente per un utilizzo efficace delle informazioni: è necessario interpretare la struttura logica del documento, ricostruirne il layout, individuare le entità rilevanti e associare i valori estratti ai rispettivi campi semantici. In questo contesto, le tecniche di \textit{Machine Learning} rivestono un ruolo chiave. Attraverso l'analisi combinata del contenuto e della struttura del documento, è possibile sviluppare sistemi in grado di riconoscere automaticamente tipologie documentali differenti e di classificare i documenti in base al loro contenuto.

L'obiettivo di questa tesi è l'implementazione di un prototipo di un sistema per il riconoscimento e la classificazione automatica di documenti tecnici in formato PDF. Un aspetto chiave del lavoro riguarda la valutazione di differenti approcci all'analisi documentale, con particolare attenzione all'accuratezza nell'estrazione dei dati, alla robustezza e alla capacità di generalizzazione su documenti eterogenei. Verrà posto l'accento sul confronto tra due approcci diversi alla risoluzione della problematica: il primo basato sull'utilizzo di servizi già esistenti sul mercato, caratterizzati da costi considerevoli e limitazioni, ed il secondo caratterizzato dall'impiego di tecnologie open source, che limitano i costi e consentono un livello maggiore di personalizzazione. 

Vengono inoltre affrontate problematiche pratiche quali la normalizzazione dei valori estratti, la gestione delle ambiguità e l'integrazione dei risultati all'interno di flussi applicativi esistenti. Tale sistema dovrà essere integrato in un'applicazione già operativa in un'azienda del settore Fashion, e sarà finalizzato ad automatizzare alcuni processi ritenuti complessi e onerosi a livello temporale per gli utenti.